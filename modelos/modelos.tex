\documentclass{article}

\usepackage{lmodern}
\usepackage{amsfonts}
\usepackage[T1]{fontenc}
\usepackage[utf8]{inputenc}
\usepackage[spanish,activeacute]{babel}
\usepackage{mathtools}

\usepackage{amsmath}
\numberwithin{equation}{section}

\title{El problema del vendedor viajero}
\setlength{\parindent}{0pt}
\begin{document}
\part{Ruta más corta}
\section{Formulación de transbordo}
\subsection{Conjuntos}
$N$ = Conjunto de nodos\\
$A$ = Conjunto de arcos
\subsection{Parámetros}
$c_{ij}$ = costo asociado al arco $(i,j)$\\
$o$ = Nodo de origen\\
$d$ = Nodo destino
\subsection{Variables}
\begin{flushleft}
\[x_{ij}={\begin{cases}1&{\mbox{si el arco $(i,j)$ se encuentra en el path}}\\0&{\mbox{en caso contrario}}\end{cases}}
\]
\end{flushleft}
\subsection{Formulación matemática}
\begin{equation}
min \sum_{(i,j) \in A} c_{ij} \cdot x_{ij}
\end{equation}
Sujeto a : \begin{align}
& \sum_{j \in N} x_{oj} = 1 \\
& \sum_{i \in N} x_{id} = 1 \\
& \sum_{(i,j) \in A} x_{ij} = \sum_{(j,h) \in A} x_{jh} &\forall j \in N \backslash \{o,d\} \\
& x_{ij} \in \{0,1\} &\forall (i,j) \in A
\end{align}




\newpage
\part{El problema del minimum spanning tree MST}
\section{Formulación de clásica}
\subsection{Conjuntos}
$N$ = Conjunto de nodos\\
$A$ = Conjunto de arcos
\subsection{Parámetros}
$c_{ij}$ = costo asociado al arco $(i,j)$
\subsection{Variables}
\begin{flushleft}
\[x_{ij}={\begin{cases}1&{\mbox{si el arco $(i,j)$ se encuentra en el árbol de expansión}}\\0&{\mbox{en caso contrario}}\end{cases}}
\]
\end{flushleft}
\subsection{Formulación matemática}
\begin{equation}
min \sum_{(i,j) \in A} c_{ij} \cdot x_{ij}
\end{equation}
Sujeto a : \begin{align}
& \sum_{(i,j) \in A} x_{ij} = |N| - 1\\
& \sum_{(i,j) \in A: i,j \in S} x_{ij} \leq |S| - 1 &\forall S \subseteq N : |S| \geq 2 \\
& x_{ij} \in \{0,1\} &\forall (i,j) \in A
\end{align}

\newpage
\section{Formulación de clásica II}
\subsection{Conjuntos}
$N$ = Conjunto de nodos\\
$A$ = Conjunto de arcos
\subsection{Parámetros}
$c_{ij}$ = costo asociado al arco $(i,j)$\\
$r$ = Nodo de origen del árbol
\subsection{Variables}
\begin{flushleft}
\[x_{ij}={\begin{cases}1&{\mbox{si el arco $(i,j)$ se encuentra en el árbol de expansión}}\\0&{\mbox{en caso contrario}}\end{cases}}
\]
\end{flushleft}
\subsection{Formulación matemática}
\begin{equation}
min \sum_{(i,j) \in A} c_{ij} \cdot x_{ij}
\end{equation}
Sujeto a : \begin{align}
& \sum_{(i,j) \in A} x_{ij} = 1 &\forall j \in N \backslash \{r\}\\
& \sum_{(i,j) \in A} x_{ij} = |N| - 1\\
& \sum_{(i,j) \in A: i,j \in S} x_{ij} \leq |S| - 1 &\forall S \subseteq N : |S| \geq 2 \\
& x_{ij} \in \{0,1\} &\forall (i,j) \in A
\end{align}

\newpage
\section{Formulación de flujo entero}
\subsection{Conjuntos}
$N$ = Conjunto de nodos\\
$A$ = Conjunto de arcos
\subsection{Parámetros}
$c_{ij}$ = costo asociado al arco $(i,j)$\\
$r$ = Nodo de origen del árbol
\subsection{Variables}
\begin{center}
\[x_{ij}={\begin{cases}1&{\mbox{si el arco $(i,j)$ se encuentra en el árbol de expansión}}\\0&{\mbox{en caso contrario}}\end{cases}}
\]
$f_{ij}$ = flujo que pasa por el arco $(i,j)$
\end{center}
\subsection{Formulación matemática}
\begin{equation}
min \sum_{(i,j) \in A} c_{ij} \cdot x_{ij}
\end{equation}
Sujeto a : \begin{align}
& \sum_{(r,j) \in A} f_{rj} = |N| - 1\\
& \sum_{(i,j) \in A} f_{ij} = 1 + \sum_{(j,h) \in A} f_{jh} &\forall j \in N \backslash \{r\} \\
& f_{ij} \leq (|N| - 1)\cdot x_{ij} &\forall (i,j) \in A\\
& \sum_{(i,j) \in A} x_{ij} = 1 &\forall j \in N \backslash \{r\}\\
& \sum_{(i,j) \in A} x_{ij} = |N| - 1\\
& \sum_{(i,j) \in A: i,j \in S} x_{ij} \leq |S| - 1 &\forall S \subseteq N : |S| \geq 2 \\
& x_{ij} \in \{0,1\} &\forall (i,j) \in A
\end{align}

\newpage
\section{Formulación de flujo multicommodity}
\subsection{Conjuntos}
$N$ = Conjunto de nodos\\
$A$ = Conjunto de arcos
\subsection{Parámetros}
$c_{ij}$ = costo asociado al arco $(i,j)$\\
$r$ = Nodo de origen del árbol
\subsection{Variables}
\begin{center}
\[x_{ij}={\begin{cases}1&{\mbox{si el arco $(i,j)$ se encuentra en el árbol de expansión}}\\0&{\mbox{en caso contrario}}\end{cases}}
\]
$f_{ij}^k$ = flujo que pasa por el arco $(i,j)$ con dirección al nodo $k$
\end{center}
\subsection{Formulación matemática}
\begin{equation}
min \sum_{(i,j) \in A} c_{ij} \cdot x_{ij}
\end{equation}
Sujeto a : \begin{align}
& \sum_{(r,j) \in A} f_{rj}^k = 1 &\forall k \in N \backslash \{r\}\\
& \sum_{(i,k) \in A} f_{ik}^k = 1 &\forall k \in N \backslash \{r\}\\
& \sum_{(i,j) \in A} f_{ij}^k = \sum_{(j,h) \in A} f_{jh}^k &\forall j,k \in N \backslash \{r\} : j \neq k \\
& f_{ij}^k \leq x_{ij} &\forall (i,j) \in A, k \in N \backslash \{r\}\\
& \sum_{(i,j) \in A} x_{ij} = 1 &\forall j \in N \backslash \{r\}\\
& \sum_{(i,j) \in A} x_{ij} = |N| - 1\\
& \sum_{(i,j) \in A: i,j \in S} x_{ij} \leq |S| - 1 &\forall S \subseteq N : |S| \geq 2 \\
& x_{ij} \in \{0,1\} &\forall (i,j) \in A
\end{align}

\newpage
\section{Formulación de Miller-Tucker-Zemblin (MTZ)}
\subsection{Conjuntos}
$N$ = Conjunto de nodos\\
$A$ = Conjunto de arcos
\subsection{Parámetros}
$c_{ij}$ = costo asociado al arco $(i,j)$\\
$r$ = Nodo de origen del árbol
\subsection{Variables}
\begin{center}
\[x_{ij}={\begin{cases}1&{\mbox{si el arco $(i,j)$ se encuentra en el árbol de expansión}}\\0&{\mbox{en caso contrario}}\end{cases}}
\]
$t_{j}$ = número de arcos entre el nodo raíz y el nodo $j$
\end{center}
\subsection{Formulación matemática}
\begin{equation}
min \sum_{(i,j) \in A} c_{ij} \cdot x_{ij}
\end{equation}
Sujeto a : \begin{align}
& \sum_{(i,j) \in A} x_{ij} = 1 &\forall j \in N \backslash \{r\}\\
& \sum_{(i,j) \in A} x_{ij} = |N| - 1\\
& t_{j} \geq t_{i} + 1 - |N| \cdot (1 - x_{ij}) &\forall (i,j) \in A, j \neq r\\
& t_{r} = 0\\
& x_{ij} \in \{0,1\} &\forall (i,j) \in A\\
& t_{j} \geq 0 &\forall j \in N
\end{align}

\newpage
\part{El problema de la p-mediana}
\section{Formulación clásica}
\subsection{Conjuntos}
$N$ = Conjunto de nodos\\
$A$ = Conjunto de arcos
\subsection{Parámetros}
$d_{ij}$ = Distancia entre el nodo de demanda $i$ y el servidor candidato $j$\\
$p$ = Cantidad de servidores a localizar
\subsection{Variables}
\begin{center}
\[x_{j}={\begin{cases}1&{\mbox{si se localiza un servidor en $j$}}\\0&{\mbox{en caso contrario}}\end{cases}}
\]
\[y_{ij}={\begin{cases}1&{\mbox{si se asigna el nodo $i$ al servidor $j$}}\\0&{\mbox{en caso contrario}}\end{cases}}
\]
\end{center}
\subsection{Formulación matemática}
\begin{equation}
min \sum_{(i,j) \in A} y_{ij} \cdot d_{ij}
\end{equation}
Sujeto a : \begin{align}
& \sum_{j \in N} x_{j} = p \\
& \sum_{j \in N} y_{ij} = 1 &\forall i \in N \\
& y_{ij} - x_{j} \leq 0 &\forall (i,j) \in A \\
& x_{j} \in \{0,1\} &\forall j \in N\\
& y_{ij} \in \{0,1\} &\forall (i,j) \in A
\end{align}

\part{El problema de la p-centro}
\section{Formulación clásica}
\subsection{Conjuntos}
$N$ = Conjunto de nodos\\
$A$ = Conjunto de arcos
\subsection{Parámetros}
$d_{ij}$ = Distancia entre el nodo de demanda $i$ y el servidor candidato $j$\\
$p$ = Cantidad de servidores a localizar
\subsection{Variables}
\begin{center}
\[x_{j}={\begin{cases}1&{\mbox{si se localiza un servidor en $j$}}\\0&{\mbox{en caso contrario}}\end{cases}}
\]
\[y_{ij}={\begin{cases}1&{\mbox{si se asigna el nodo $i$ al servidor $j$}}\\0&{\mbox{en caso contrario}}\end{cases}}
\]
$W$ = Distancia máxima entre un nodo de demanda y su servidor asignado
\end{center}
\subsection{Formulación matemática}
\begin{equation}
min \; W
\end{equation}
Sujeto a : \begin{align}
& \sum_{j \in N} x_{j} = p \\
& \sum_{j \in N} y_{ij} = 1 &\forall i \in N \\
& y_{ij} - x_{j} \leq 0 &\forall (i,j) \in A \\
& W - \sum_{j \in N} d_{ij} \cdot y_{ij} \geq 0 &\forall i \in N\\
& x_{j} \in \{0,1\} &\forall j \in N\\
& y_{ij} \in \{0,1\} &\forall (i,j) \in A\\
& W \geq 0
\end{align}



\newpage
\part{El problema del maximal covering}
\section{Formulación clásica}
\subsection{Conjuntos}
$N$ = Conjunto de nodos\\
$A$ = Conjunto de arcos
\subsection{Parámetros}
$d_{ij}$ = Distancia entre el nodo de demanda $i$ y el servidor candidato $j$\\
$h_{i}$ = Demanda del nodo $i$\\
$S$ = Radio de cobertura\\
$p$ = Cantidad de servidores a localizar\\
$C_{i}$ = $\{j \mid d_{ij} \leq S\}$
\subsection{Variables}
\begin{flushleft}
\[x_{j}={\begin{cases}1&{\mbox{si se localiza el servidor en $j$}}\\0&{\mbox{en caso contrario}}\end{cases}}
\]
\[y_{i}={\begin{cases}1&{\mbox{si la demanda del nodo $i$ es cubierta}}\\0&{\mbox{en caso contrario}}\end{cases}}
\]
\end{flushleft}
\subsection{Formulación matemática}
\begin{equation}
max \sum_{i \in N} y_{i} \cdot h_{i}
\end{equation}
Sujeto a : \begin{align}
& \sum_{j \in C_{i}} x_{j} - y_{i} \geq 0 &\forall i \in N \\
& \sum_{j \in N} x_{j} = p\\
& x_{j} \in \{0,1\} &\forall j \in N\\
& y_{j} \in \{0,1\} &\forall j \in N
\end{align}




\newpage
\part{El problema del set-covering}
\section{Formulación clásica}
\subsection{Conjuntos}
$N$ = Conjunto de nodos\\
$A$ = Conjunto de arcos
\subsection{Parámetros}
$d_{ij}$ = Distancia entre el nodo de demanda $i$ y el servidor candidato $j$\\
$S$ = Radio de cobertura\\
$C_{i}$ = $\{j \mid d_{ij} \leq S\}$
\subsection{Variables}
\begin{flushleft}
\[x_{j}={\begin{cases}1&{\mbox{si se localiza el servidor en $j$}}\\0&{\mbox{en caso contrario}}\end{cases}}
\]
\end{flushleft}
\subsection{Formulación matemática}
\begin{equation}
min \sum_{j \in N} x_{j}
\end{equation}
Sujeto a : \begin{align}
& \sum_{j \in C_{i}} x_{j} \geq 1 &\forall i \in N \\
& x_{j} \in \{0,1\} &\forall j \in N
\end{align}


\newpage
\part{El problema del vendedor viajero}
\section{Formulación de Flujo Entero}
\subsection{Conjuntos}
$N$ = Conjunto de nodos\\
$A$ = Conjunto de arcos
\subsection{Parámetros}
$c_{ij}$ = costo asociado al arco $(i,j)$
$d$ = nodo de origen
\subsection{Variables}
\begin{flushleft}
\[x_{ij}={\begin{cases}1&{\mbox{si el arco $(i,j)$ se encuentra en el tour}}\\0&{\mbox{en caso contrario}}\end{cases}}
\]
\[f_{ij} = \mbox{flujo enviado desde el nodo i, hacia el nodo j}\]
\end{flushleft}
\subsection{Formulación matemática}
\begin{equation}
min \sum_{(i,j) \in A} c_{ij} \cdot x_{ij}
\end{equation}
Sujeto a : \begin{align}
& \sum_{(i,j) \in A} x_{ij} = 1 &\forall j \in N \\
& \sum_{(i,j) \in A} x_{ij} = 1 &\forall i \in N \\
& \sum_{(d,j) \in A} f_{dj} = |N| - 1 \\
& \sum_{(i,j) \in A} f_{ij} - \sum_{(j,h) \in A} f_{jh} = 1 &\forall j \in N \backslash \{d\} \\
& f_{ij} \leq (|N| - 1) \cdot  x_{ij} & \forall (i,j) \in A\\
& x_{ij} \in \{0,1\} &\forall (i,j) \in A\\
& f_{ij} \geq 0 &\forall (i,j) \in A
\end{align}

\newpage
\section{Formulación MTZ}
\subsection{Conjuntos}
$N$ = Conjunto de nodos\\
$A$ = Conjunto de arcos
\subsection{Parámetros}
$c_{ij}$ = costo asociado al arco $(i,j)$
$d$ = nodo de origen
\subsection{Variables}
\begin{flushleft}
\[x_{ij}={\begin{cases}1&{\mbox{si el arco $(i,j)$ se encuentra en el tour}}\\0&{\mbox{en caso contrario}}\end{cases}}
\]
\[t_{i} = \mbox{posicion en que se recorre el nodo i en el tour}\]
\end{flushleft}
\subsection{Formulación matemática}
\begin{equation}
min \sum_{(i,j) \in A} c_{ij} \cdot x_{ij}
\end{equation}
Sujeto a : \begin{align}
& \sum_{(i,j) \in A} x_{ij} = 1 &\forall j \in N \\
& \sum_{(i,j) \in A} x_{ij} = 1 &\forall i \in N \\
& t_{j} \geq t_{i} + 1 - |N| \cdot (1 - x_{ij}) & \forall (i,j) \in A, j \neq d\\
& t_{d} = 0\\
& x_{ij} \in \{0,1\} &\forall (i,j) \in A\\
& t_{i} \in \mathbb{Z}^{+}_{0} &\forall i \in N
\end{align}



\end{document}


