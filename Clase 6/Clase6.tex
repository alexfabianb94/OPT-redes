\documentclass[letterpaper]{article}
\usepackage[T1]{fontenc}
\usepackage[utf8]{inputenc}
\usepackage[spanish,es-tabla]{babel}
\usepackage[lmargin=3cm,rmargin=3cm,top=3cm,bottom=3cm]{geometry}
\usepackage{amsmath,amsfonts}
\usepackage{mathtools}
\usepackage{graphicx}
\usepackage{fancyhdr}
\usepackage{caption}


\lhead{Universidad del Bío-Bío\\ Facultad de ingeniería \\ Escuela de Ingeniería Industrial}
\rhead{\begin{picture}(0,0) \put(-75,0){\includegraphics[height=15mm]{logos/escudo_ubb_2}} \end{picture}}

\pagestyle{fancy}


\usepackage{titlesec}
\titleformat*{\section}{\large\bfseries\centering}
\titleformat*{\subsection}{\normalsize\bfseries}

\begin{document}
\vspace*{0.1\baselineskip}
\section*{Práctica 6}
\subsection*{Problema 1}
Considere el grafo $G=(N,A)$, donde $N$ es el conjunto de nodos y $A$ el conjunto de arcos. El parámetro $d_{ij}$ determina la distancia entre los nodos $i$ y $j$, con $(i,j) \in A$.

\begin{itemize}
\item Formule un modelo de RMC, considerand $s$ como el nodo de origen y $t$ como el nodo de destino, $s,t \in N$
\item Formule un modelo para el MST
\item Formule un modelo para la p-mediana
\item Formule un modelo para el maximal covering, con radio de cobertura $S$
\item Formule un modelo para el set-covering, con radio de cobertura $S$
\end{itemize}

\end{document}
